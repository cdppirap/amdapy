\documentclass[a4paper,11pt]{report}

\author{Schulz Alexandre}
\title{Package documentation\\AMDAPy}

\begin{document}
\maketitle

\section{REST}
\subsection{Errors}
Rest errors are defined in the \emph{amdapy.rest.errors} module.

\begin{itemize}
	\item AuthenticationError : raised when requesting an authentication token fails
	\item CollectionRetrievalError : failed to get dataset collection
	\item DatasetDownloadError : failed to download dataset
	\item LargeTimePeriodError : requested time period was too large
	\item ParameterDownloadError : failed to download parameter

		% status error
	\item ProcessStatusError : failed to get status for process

\end{itemize}

\section{amdapy functionalities}

List here basic functions that are within the scope of \emph{amdapy} : 
\begin{itemize}
	\item Download public datasets, parameters from AMDA
	\item Download private parameters from user space
	\item Resample dataset (defaults to original sampling time)
	\item Data structures available : 
	\begin{itemize}
		\item Datasets
		\item Parameters
		\item Time tables
		\item Catalogs
	\end{itemize}
	\item Explore AMDA database, search criterion : 
		\begin{itemize}
			\item mission
			\item instrument
			\item start and stop time
			\item units
			\item observed region
			\item measurement type
			\item more, ...
		\end{itemize}
	\item Retrieve SPASE metadata pretaining to mission, instruments, people, datasets
	\item Full documentation
	\item Jupyter notebook
\end{itemize}

\newpage

\paragraph{List of objects} : 
\begin{itemize}
	\item amdapy.Mission(id=None)
	\begin{itemize}
		\item id : name of the mission, it is unique within AMDA
	\end{itemize}
	\item amdapy.Instrument(id=None, mission=None)
	\begin{itemize}
		\item id : name of the instrument, unique within its corresponding mission
	\end{itemize}
	\item amdapy.AmdaDataFrame(id=None)
	\begin{itemize}
		\item has a \emph{pandas\_df} method that returns the content of the datastructure in a dataframe object
		\item can specify the start and stop times, sampling rate, ...
	\end{itemize}
	\item amdapy.Dataset(AmdaDataFrame, parameters=[])
	\begin{itemize}
		\item contains a list of parameters
	\end{itemize}
	\item amdapy.Parameter(AmdaDataFrame)
	\begin{itemize}
		\item has id, units, name, shape
	\end{itemize}
	\item amdapy.TimeTable(AmdaDataFrame)
	\item amdapy.Catalog(AmdaDataFrame)
\end{itemize}

\paragraph{Iterators} : 
\begin{itemize}
	\item amdapy.datasets(mission=None, instrument=None)
	\item amdapy.missions() 
	\item amdapy.instruments(mission=None)
	\item amdapy.parameters(mission=None, instrument=None, dataset=None)
	\item amdapy.derived\_parameters(userid=None, password=None)
\end{itemize}

\section{amdapy.Mission}

In AMDA all mission are uniquely identified by their name, which is stored in the Mission.id attribute.

Missions have SPASE

\section{The AMDAPy package}
The aim of this package is to propose a easy to use python interface for accessing data stored on
the AMDA plateform (and similar plateforms), applying common transformation and visualizing data.

Data stored on AMDA is mostly used for astronomic and plasma physics, as such some functions and
visualizing methods can be very specific to a certain used case.

We would like to offer a collection of well documented functions that can be used by researchers to
rapidly test known models and transformation on new datasets and visualize the results. Ideally the 
output plots should be as close to possible to publication quality.

When working on a new subject concerning dataset X a researcher may produce a variety of plots
and tranformation functions. When the results finaly appear on paper it is not always easy to
reproduce the results. The plateform should offer a centralized and safe way of storing papers
and the simulation codes used for producing the results.

Lets try to describe the structure of the plateform such that we imagine using it. The plateform 
can be installed locally on the users machine by installing the amdapy package, otherwise we imagine 
having a web interface that can be accessed by the public.

When used locally : user can create a local session for storing data, apply transformation and plotting 
functions and store the results on local machine. The user can search for datasets (by name, by type of 
content), transformations and plots. The last point reveals the importance of describing the data. 
Dataset can be described using various norms (SPASE, ...), the plateform should be able to extract the
most important information. It is this information that allows us to determine if the dataset is
compatible with the use of certain transformation functions. For example we might want to define a 
transformation that computed coordinate values in coordinate system B from data expressed in 
coordinate system A. The input data must verify some conditions : the data should have 3 spatial 
dimensions, if it is not the case then the dataset is not compatible with the transformation. These
notions will need to be further defined in the rest of this document.

Dataset important components : 
  variables : datatype, name, fillvalue

\subsection{Used locally}
It is common for researchers to retrive data they are interested in to their local machine and
perform the desired task locally. As it is currently implemented AMDA only allows visualizing 
data from within the plateform itself. This creates some complications : data is retrieved for 
each plot request which can be time consumming for large datasets. To address this issue the user
of the API can load a specific dataset to a local storage structure allowing much faster data 
retrieval. Additionnaly the user can then define transformations and visualization function tailored
to his specific needs. 

\subsection{Web interface}
The plateform should be accessible and editable through a web interface. The interface is composed
of a dataset tree (containing datasets from various providers), a transformation tree (containing user
defined functions that can be used on the data), a plot tree (containing the user defined plotting 
functions), etc... (notebook tree, contains user defined notebooks).

The user can then apply transformation and execute plotting. Code is executed on the server ? or on the
local machine (need javascript). The user can see and edit the code of the transformation and plotting functions to adapt them to the researchers use case. Once edited the new function is saved to a user 
specific database. Functions defined this way can then be retrived through the API call (with user 
authentification).




\subsection{Package structure}
The package is composed of a number of subpackages that have various functionalities:
\begin{itemize}
\item core : functions and classes used throughout the project (base dataset representation, dataset
identification, time manipulation, etc...)
\item vi : functions and classes for manipulation Virtual Instruments (mostly used by the database administrator for adding datasets to AMDA.
\item plots : functions and classes for visualizing the data.
\item transform : functions and classes for transforming the data (Machine Learning module will be stored here)
\item scripts : collection of scripts.
\item db : collection of classes for managing the databases the packages needs.
\end{itemize}

The project will be composed of a variety of databases that each have different function : 
\begin{itemize}
\item AMDA database : the main source of datafiles.
\item Session database : a local database set up on the users machine allowing him to have access to
the data even without an internet access.
\item Transformation database : a database containing transformation functions that the used can query 
to retrieve commonly used functions.
\item Plot database : similar to the transformation database but containing plotting functions.
\end{itemize}

\section{Data providers}

\section{Transformations}

\section{Plotting}
\end{document}
